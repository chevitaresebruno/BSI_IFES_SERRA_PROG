\captionsetup{justification=centering,margin=0cm}

%inicio do capitulo
\chapter[INFORMAÇÕES SOBRE O TRABALHO]{INFORMAÇÕES SOBRE O TRABALHO}

Esta seção apenas existe para atender às especificidades das entregas parciais deste trabalho. Esta seção é dinâmica e existem observações que apenas aplicam-se a essa entrega, portanto nas estregas seguintes poderá haver modificação do texto aqui exposto, bem como da Tabela a seguir.

\begin{table}[htbp]

\begin{tabularx}{\textwidth}{X|C|C|C|C|C|C}
    \hline
        
    \textbf{Entregas} & \textbf{Atividades} & \textbf{Data} & \multicolumn{4}{c}{\textbf{Entregue no Prazo?}} \\ \hline
    Entrega 1 & 1, 2, 3 e 4 & 18/05/2024 & x & x & x & x \\\hline
    Entrega 2 & anterior + 5 e 6 & 29/05/2024 & \multicolumn{2}{c|}{x} & \multicolumn{2}{|c}{} \\\hline
    Entrega 3 & anterior + 7 e 8 & 05/06/2024 &  \multicolumn{4}{c}{\textbf{Controle feito pelo AVA}} \\

    \hline
    
\end{tabularx}
\end{table}

Observações sobre esta entrega: por algum motivo o overleaf, ferramenta que utilizamos para criar o pdf em LATEX, não conseguiu as imagens do carro.svg. Queríamos usar essas imagens para melhorar o fluxo de leitura do trabalho, mas o compilador simplesmente não executa. Assim, as FIGURAS 7.1 e 7.3 estão com o pinguinzin fofin do Linux. Tentaremos, para a próxima entrega, utilizar um editor local do LATEX para gerar o PDF e então estar com a imagem certa. Outra coisa importante é sobre a tabela 9.1; ela ficou gigantesca no final dela os números são muito grandes, então algumas casas ficaram para fora da limitação da célula. Nós sabemos da existência do erro mas não conseguimos corrigir, mas esse erro não estará na próxima entrega. Por fim é possível que existam alguns error de português no texto; como ele ainda não passou pela revisão final é que a leitura ainda esteja um pouco travada e que erros de português existam, contudo acreditamos que o texto encontra-se suficientemente legível.