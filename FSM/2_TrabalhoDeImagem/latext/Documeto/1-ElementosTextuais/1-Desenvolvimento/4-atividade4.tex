\captionsetup{justification=centering,margin=0cm}
\label{cap:atividade4}  % Forma de referenciar o capítulo no comando \ref

%inicio do capitulo
\chapter[Atividade 4: COMPARANDO GIF COM PNG-8]{Atividade 4: COMPARANDO GIF COM PNG-8}

Nem só de um milhão de cores vive um homem. Houve uma época em que 8-bits eram suficiente para fazer imagens, músicas e jogos (inclusive muitos deles eram melhores do que os que saem hoje em dia). Assim, decidimos colocar um pé no passado e observar como o GIF e o PNG-8 comportam-se.

\paragrafo Após realizarmos a conversão, pode-se ter impressão de que as fotos estavam desbotadas. É como se elas tivessem perdido parte da cor.. Ah pera, isso na realidade ocorreu né. No geral, o png teve uma taxa de compressão melhor no geral, com pouco mais de 34,3 MB, enquanto as imagens do gif somam pouco mais de 39,4 MB. Se comparadas entre si, as imagens gif e png possuem pouca diferença visual entre si, parecem a mesma imagem, apesar de serem diferentes.

\paragrafo Contudo, fizemos novos testes, mas agora reduzido as imagens originais para 10\% de seu tamanho original. Além das cores, aqui as diferenças entre as imagens em GIF e PNG são bem mais nítidas. Para iniciar, o PNG comprimiu as imagens para um total 1,41 MB, enquanto o GIF para míseros 712 KB, uma diferença considerável. Contudo, o PNG possui imagens com qualidades melhores. Em regiões com cores muito semelhantes, é perceptível a presença de uma espécie de ruído na imagem, como uma inteferência na televisão quando o sinal está ruim. Já no PNG esse ruído também faz-se presente, mas de forma reduzida e melhor mascarada.