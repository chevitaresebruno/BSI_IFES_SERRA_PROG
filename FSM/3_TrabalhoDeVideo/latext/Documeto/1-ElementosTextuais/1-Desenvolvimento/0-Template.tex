\captionsetup{justification=centering,margin=0cm}
\label{cap:REF}  % Forma de referenciar o capítulo no comando \ref

%inicio do capitulo
\chapter[TÍTULO]{TÍTULO}
\index{TÍTULO}



\section{TABELAS NA ABNT}
% MODELOS DE TABELAS NA ABNT

\begin{table}[htbp]
\centering

\caption{Título da tabela}
\label{tab:Tabela1}

\begin{tabularx}{\textwidth}{XCCC}
    \hline
        
    \multicolumn{1}{c}{\textbf{Cabeçalho}} &  \textbf{Cabeçalho} & \textbf{Cabeçalho} & \textbf{Cabeçalho} \\ \hline

    Linha 1 & Linha 1 & Linha 1 & Linha 1 \\
    Linha 2 & Linha 2 & Linha 2 & Linha 2 \\

    \hline
    
\end{tabularx}

\autoriaPropria

\end{table}

Referenciado a Tabela \ref{tab:Tabela1}

\begin{table}[htbp]

\begin{tabularx}{\textwidth}{XCCC}
    \hline
        
    \multicolumn{1}{c}{\textbf{Cabeçalho}} & \textbf{Cabeçalho} & \textbf{Cabeçalho} & \textbf{Cabeçalho} \\ \hline

    \multirow{2}{*}{Linhas 1 e 2} & Linha 1 & Linha 1 & Linha 1 \\
     & Linha 2 & Linha 2 & Linha 2 \\

    \hline
    
\end{tabularx}
\end{table}


% ATENÇÃO PARA O USO DOS COMANDOS \footnotemark e \footnotetext{}
\begin{table}[htbp]
\centering

\begin{tabularx}{\textwidth}{XCCC}
    \hline
        
    \multicolumn{1}{c}{\textbf{Cabeçalho}} & \textbf{Cabeçalho} & \textbf{Cabeçalho} & \textbf{Cabeçalho} \\ \hline

    \multirow{2}{8em}{Se precisar de linhas horizontais\footnotemark} & Linha 1 & Linha 1 \\ \hhline{~---}
     & Linha 2 & Linha 2 & Linha 2 \\

    \hline
    
\end{tabularx}
\end{table}
\footnotetext{Para esses comandos funcionarem, coloque o footnotetext fora da tabela}

\section{Texto}
TEXTO NORMAL

\paragrafo Espaçamento de Parágrafo\index{Parágrafo}