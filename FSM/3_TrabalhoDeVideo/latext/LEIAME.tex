% COMO USAR ESSE MODELO DE TCC

% Inicie abrindo o arquivo de configurações do documento, no caminho Configurações/0-SobreOTrabalho.tex
% Altere todas as informações necessárias, mas não mexa na seção de informações Fixas.
% Ainda nessa pasta, existe o arquivo 1-VariaveisEComandosUteis.tex. Nele você pode colocar comandos e variáveis próprias à vontade para usar no documento. Uma dica, já existe o comando \autoriaPropria. Use ele sempre que usar um quadro, tabela ou figura que seja de sua autoria.

% Agora, vá ao arquivo main.tex. É nele que você vai adicionar ou editar as informações que aparecem no documento. Pesquise pela seção DOCUMENTO. Logo acima existe a importação dos arquivos de configuração. Se quiser que as tabelas, figuras e quadros do seu documento não respeitem a ordem dos capítulos, exclua essa linha.
% Vá até a seção ELEMENTOS PRÉ-TEXTUAIS. As três seções subsequentes são importações dos arquivos que devem aparecer no documento ao lado. Configure quais documentos você quer que aparecam. No caso dos textos de desenvolvimento, coloquê-os na seção DESENVOLVIMENTO.

% Agora, abra a pasta Documento. Nela estão separados todos os arquivos por sua categoria. Escreva naqueles que você for utilizar e seja feliz.
% Para adicionr textos de desenvolvimento, vá na pasta 1-ElementosTextuais/1-Desenvolvimento e adicione os capítulos nessa pasta.
% Se você estiver na pasta 1-ElementosTextuais/1-Desenvolvimento, verá que existe um arquivo chamado 0-Template.tex. Ele contém algumas informações úteis para iniciar um novo capítulo e informações sobre tabelas. Você pode ver esse texto no capítulo 6 na aba de renderização do Overleaf.

% Se for usar referências, acesse o arquivo references.bib e siga os templates.