\captionsetup{justification=centering,margin=0cm}

%inicio do capitulo
\chapter[ÁUDIOS BINAURAIS (ATIVIDADE 2)]{ÁUDIOS BINAURAIS (ATIVIDADE 2)}
\index{ÁUDIOS BINAURAIS (ATIVIDADE 2)}

A audição humana é composta por diferentes camadas de complexidade. Como o cérebro funciona por meio de uma vasta ligação entre neurônios, um som pode despertar uma memória ou mesmo um sentimento, além de as diferenças entre os canais de som proporcionarem experiências diferentes e mais imersivas em alguns contextos.

\hspace{1.5 cm} O objetivo de um áudio binaural é proporciona a sensação de profundidade e imersão a partir de diferentes sons que são emitidos em intensidades diferentes, ou mesmo frequências diferentes, nos dois ouvidos. É importante compreender isso, pois, essa questão da \textit{mixagem} do som influencia diretamente na compressão e manipulação de formatos destes.

\hspace{1.5 cm} Essa seção é empírica e busca evidenciar as percepções dos autores deste documento de modo individual. Não existe, necessariamente, um estudo aprofundado ou científico dessas análises.

\section{Percepções de Bruno da Fonseca Chevitarese}
Um áudio binaural, como exposto anteriormente, consiste particularmente na sensação de imersão e profundidade. Como esse tipo de áudio não está associado a algo em particular, como sons da natureza, ou outras categorias, entendo que as manifestações corporais dependem de cada indivíduo. Assim, para manter um controle das observações, ater-me-ei a analisar as minhas reações em relação aos áudios propostos nas especificações do trabalho.

\hspace{1.5 cm} A partir da análise dos três áudios pude reparar que não sou muito expressivo. Mantive-me na maior parte do tempo neutro em relação aos sons, eles não me incomodavam nem afetavam de alguma forma. Contudo, sou uma pessoa com uma mente bastante imaginativa; o que eu não percebia no corpo, percebia na mente.

\hspace{1.5 cm} Reforço novamente que o áudio binaural proporciona a sensação de profundidade e imersão e em cada áudio eu me imaginava na cena. No primeiro áudio, eu percebia a movimentação do papel e tentava entender no quê o papel estava sendo "transformado", ou seja, como era amassado. Quanto ao segundo áudio tentei rastrear as bolas, compreender da onde elas vinham, a altura delas até cair no chão e a posição onde colidiram. Por fim, no terceiro áudio tentei imaginar o ambiente, compreendendo da onde as pessoas falavam e como movimentava-se pela cena. Acredito que uma parte da imersão tenha se quebrado, pois, meu inglês não é muito rebuscado, então eu tentava entender as falas sem necessariamente entendê-las.

\hspace{1.5 cm} Acredito que a minha experiência decorre das minhas experiência particulares. Eu sempre fui uma pessoa que tem a necessidade de entender exatamente o que está acontecendo e não gosto de ser pego de surpresa, então tento extrair a maior quantidade de informações possíveis do ambiente e dos corpos que o compõe

\section{Percepções de Luiz Felipe Elizeta}

Ao estar constantemente exposto à ideia de áudio binaural, adquiri um conhecimento íntimo das minhas reações a ele. Experimento arrepios e um leve sobressalto, percebo minhas emoções amplificadas, e sinto-me mais alerta e concentrado enquanto ouço. Parece quase como se estivesse sendo constantemente ludibriado, porém, paradoxalmente, também me sinto profundamente influenciado por ele, dependendo do contexto, como se fosse o principal catalisador das minhas reações instintivas.

\hspace{1.5 cm} No entanto, estes são sentimentos que já experienciei antes de estudar multimídia. Agora, descubro a existência de um áudio projetado para melhorar a concentração e o foco, e decido testá-lo. Surpreendentemente, os resultados foram bastante positivos. Durante os períodos de concentração, o áudio simplesmente se fundia ao fundo, e nos momentos mais relaxados, qualquer ruído ambiente era suprimido, permitindo-me imergir ainda mais na atividade em que estava envolvido.

\hspace{1.5 cm} Assim, acredito que algo semelhante ocorra durante o estudo: com o tempo, o áudio se torna imperceptível e os ruídos externos parecem diminuir, proporcionando um ambiente propício para a concentração.
