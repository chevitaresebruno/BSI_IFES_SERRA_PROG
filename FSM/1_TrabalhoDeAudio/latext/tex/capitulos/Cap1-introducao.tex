\captionsetup{justification=centering,margin=0cm}

%inicio do capitulo
\chapter[INTRODUÇÃO]{INTRODUÇÃO}
\index{Introdução}

A tecnologia é de suma importância para o ser humano contemporâneo. Os Hominídios da pedra lascada, de certo, teriam vergonha de seus progenitores ao verem a dificuldade para abrir um simples pote de picles ou esquentar comidas no micro-ondas. Apesar das vantagens e conveniências proporcionadas pelo avanço tecnológico, a biologia segue seu curso independentemente das vontades humanas. Tanto nossos antecessores quanto nós, \textit{Homo sapiens}, compartilhamos estruturas de percepção da realidade bastante similares.

\hspace{1.5 cm} Com o advento das sociedades humanas, muitas necessidades primitivas, como a caça, tornaram-se obsoletas. No entanto, a necessidade de compreender o mundo ao nosso redor permanece crucial. Na era da abundância de tempo livre e da busca por estímulos, a multimídia surge como o \textit{Titanic} num mar de tédio, apenas esperando o próximo \textit{Iceberg} aparecer para naufragar. 

\hspace{1.5 cm} A multimídia, como o nome sugere, engloba diferentes formas de estímulo, principalmente no contexto de áudio e imagem. Essas formas de mídia são amplamente utilizadas em diversos contextos, como jogos, vídeos e fotografias. No entanto, dois desafios principais surgem: a representação de informações lógicas em formato digital e o espaço ocupado por essas mídias. Este trabalho, concentra-se no estudo do armazenamento de informações, com ênfase na compressão de arquivos e manipulação de formatos e seu impacto no contexto da multimídia, especialmente no âmbito do áudio.