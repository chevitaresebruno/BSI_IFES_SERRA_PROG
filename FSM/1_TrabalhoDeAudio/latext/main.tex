\documentclass[times,english,brazil,oneside,a4paper,fleqn]{ifes8}

\usepackage[utf8]{inputenc}
\usepackage{lastpage}           
\usepackage[alf]{abntex2cite}
\usepackage{microtype}          
\usepackage{morefloats}        
\usepackage{listings}
\usepackage{xcolor}
\usepackage{tikz}
\usepackage{hyperref}
\usetikzlibrary[topaths]
\usepackage{mathtools}
\usepackage[final]{pdfpages}
\usepackage{caption}
\usepackage{subcaption}
\usepackage{multirow}
\usepackage{boxhandler}
\usepackage[normalem]{ulem}
\usepackage{dirtytalk}
\usepackage{float}
\usepackage{siunitx}
\usepackage{svg}
\usepackage[none]{hyphenat}
\usepackage{breqn}
\usepackage{upgreek}
\usepackage{amsmath}
\hyphenation{es-ta-be-le-ci-men-to}
\raggedbottom
\selectlanguage{brazil}
\setlength{\abovecaptionskip}{5pt}


%%% Define que todos os códigos fontes construídos com o ambiente
%%% `lstlisting' terão uma borda simples.
\lstset{ %
  backgroundcolor=\color{white},   % choose the background color
  basicstyle=\footnotesize,        % size of fonts used for the code
  breaklines=true,                 % automatic line breaking only at whitespace
  captionpos=b,                    % sets the caption-position to bottom
  commentstyle=\color{mygreen},    % comment style
  escapeinside={\%*}{*)},          % if you want to add LaTeX within your code
  keywordstyle=\color{blue},       % keyword style
  stringstyle=\color{mymauve}, 
  numbers=left,
  stepnumber=1,    
  firstnumber=1,
  numberfirstline=true% string literal style,
   frame=top,frame=bottom, frame=single,
   captionpos=t,
   literate=
  {á}{{\'a}}1 {é}{{\'e}}1 {í}{{\'i}}1 {ó}{{\'o}}1 {ú}{{\'u}}1
  {Á}{{\'A}}1 {É}{{\'E}}1 {Í}{{\'I}}1 {Ó}{{\'O}}1 {Ú}{{\'U}}1
  {à}{{\`a}}1 {è}{{\`e}}1 {ì}{{\`i}}1 {ò}{{\`o}}1 {ù}{{\`u}}1
  {À}{{\`A}}1 {È}{{\'E}}1 {Ì}{{\`I}}1 {Ò}{{\`O}}1 {Ù}{{\`U}}1
  {ä}{{\"a}}1 {ë}{{\"e}}1 {ï}{{\"i}}1 {ö}{{\"o}}1 {ü}{{\"u}}1
  {Ä}{{\"A}}1 {Ë}{{\"E}}1 {Ï}{{\"I}}1 {Ö}{{\"O}}1 {Ü}{{\"U}}1
  {â}{{\^a}}1 {ê}{{\^e}}1 {î}{{\^i}}1 {ô}{{\^o}}1 {û}{{\^u}}1
  {Â}{{\^A}}1 {Ê}{{\^E}}1 {Î}{{\^I}}1 {Ô}{{\^O}}1 {Û}{{\^U}}1
  {œ}{{\oe}}1 {Œ}{{\OE}}1 {æ}{{\ae}}1 {Æ}{{\AE}}1 {ß}{{\ss}}1
  {ű}{{\H{u}}}1 {Ű}{{\H{U}}}1 {ő}{{\H{o}}}1 {Ő}{{\H{O}}}1
  {ç}{{\c c}}1 {Ç}{{\c C}}1 {ø}{{\o}}1 {å}{{\r a}}1 {Å}{{\r A}}1
  {€}{{\EUR}}1 {£}{{\pounds}}1
}

\newcommand{\ifestex}{\textsf{Ifes$8$}}


% Informação aqui
% Preencher
% Informações sobre o trabalho

\titulo{Um Estudo das Consequências da Compressão de Arquivos Multimídia em Diferentes Contextos de Áudio}
\autor{Bruno da Fonseca Chevitarese e Luiz Felipe Elizeta}
\autorficha{}
\local{Serra}
\data{2023}
\orientador[Orientador:]{Prof. Dr. Flávio Giraldeli}
\instituicao{Instituto Federal de Educação, Ciência e Tecnologia do Espírito Santo}
\curso{Curso Superior de Bacharelado em Sistemas de Informação}
\tipotrabalho{Trabalho de Acompanhamento de Disciplina}
\preambulo{Este documento tem por objetivo cumprir com o acordo pré-estabelecido de avaliações da disciplina Fundamentos de Sistema Multimídia, lecionada pelo professor Flávio Giraldeli}


% ----------------------------------------------------------
% Documento
% ----------------------------------------------------------

\begin{document}
\setsecnumformat{\csname the#1\endcsname\space}
\renewcommand{\afterchapternum}{\hspace{-4pt}}

\imprimircapa
\addtocounter{page}{-1}

\imprimirfolhaderosto*


\addtocounter{page}{-1}


% ----------------------------------------------------------
% FORMALIDADES
% ----------------------------------------------------------

\renewcommand{\afterloftitle}{\null\\[5mm]}
\renewcommand{\afterlottitle}{\null\\[5mm]}
\renewcommand{\afterloqtitle}{\null\\[5mm]}
\renewcommand{\aftertoctitle}{\null\\[5mm]}


%%% ====================================================================
%%% Lista de figuras
\renewcommand{\listfigurename}{LISTA DE FIGURAS}
\pdfbookmark[0]{\listfigurename}{lof}
\listoffigures*
\cleardoublepage


%%% ====================================================================
%%%Lista de tabelas
    \renewcommand{\listtablename}{LISTA DE TABELAS}
    \pdfbookmark[0]{\listtablename}{lot}
    \listoftables*
    \cleardoublepage



%%% ====================================================================
%%% Lista de quadros
\pdfbookmark[0]{\listadequadrosname}{loq}
\listadequadros*
\cleardoublepage


%%% ====================================================================
%%% Lista de siglas
\begin{siglas}
    \simb{Ifes}       Instituto Federal de Ensino, Ciências e Tecnologia do Espírito Santo 

    \simb{PA} Progressão Aritmética
    \simb{PG} Progressão Geométrica
    \simb{MB} Mega Bytes
\end{siglas}

\cleardoublepage


%%% ====================================================================
%%% Sumário --- Table of Contents
\renewcommand{\contentsname}{SUMÁRIO}
\pdfbookmark[0]{\contentsname}{toc}
\tableofcontents*
\cleardoublepage


% ----------------------------------------------------------
% ELEMENTOS TEXTUAIS
% ----------------------------------------------------------
\textual
\captionsetup{justification=justified,singlelinecheck=false}

\captionsetup{justification=centering,margin=0cm}

%inicio do capitulo
\chapter[APRESENTAÇÃO]{APRESENTAÇÃO}
\index{APRESENTAÇÃO}
Este documento tem por objetivo desenvolver os conhecimentos da teoria científica na prática da realidade. A partir da coleta de dados obtidos através de múltiplos testes e sobre a égide dos aprendizados da sala de aula, o trabalho buscará avaliar as influências de diferentes formas de compressão em arquivos multimídia em diferentes contextos. A multimídia analisada nesse trabalho são arquivos de textos (baseados em geração numérica) e arquivos de áudio.

\captionsetup{justification=centering,margin=0cm}

%inicio do capitulo
\chapter[INTRODUÇÃO]{INTRODUÇÃO}
\index{Introdução}

A tecnologia é de suma importância para o ser humano contemporâneo. Os Hominídios da pedra lascada, de certo, teriam vergonha de seus progenitores ao verem a dificuldade para abrir um simples pote de picles ou esquentar comidas no micro-ondas. Apesar das vantagens e conveniências proporcionadas pelo avanço tecnológico, a biologia segue seu curso independentemente das vontades humanas. Tanto nossos antecessores quanto nós, \textit{Homo sapiens}, compartilhamos estruturas de percepção da realidade bastante similares.

\hspace{1.5 cm} Com o advento das sociedades humanas, muitas necessidades primitivas, como a caça, tornaram-se obsoletas. No entanto, a necessidade de compreender o mundo ao nosso redor permanece crucial. Na era da abundância de tempo livre e da busca por estímulos, a multimídia surge como o \textit{Titanic} num mar de tédio, apenas esperando o próximo \textit{Iceberg} aparecer para naufragar. 

\hspace{1.5 cm} A multimídia, como o nome sugere, engloba diferentes formas de estímulo, principalmente no contexto de áudio e imagem. Essas formas de mídia são amplamente utilizadas em diversos contextos, como jogos, vídeos e fotografias. No entanto, dois desafios principais surgem: a representação de informações lógicas em formato digital e o espaço ocupado por essas mídias. Este trabalho, concentra-se no estudo do armazenamento de informações, com ênfase na compressão de arquivos e manipulação de formatos e seu impacto no contexto da multimídia, especialmente no âmbito do áudio.
\captionsetup{justification=centering,margin=0cm}

%inicio do capitulo
\chapter[COMPRESSÃO DE ARQUIVOS (ATIVIDADE 1, ENTROPIA NA PRÁTICA)]{COMPRESSÃO DE ARQUIVOS (ATIVIDADE 1, ENTROPIA NA PRÁTICA)}
\index{COMPRESSÃO DE ARQUIVOS (ATIVIDADE 1, ENTROPIA NA PRÁTICA)}
Conforme estabelecido na introdução do trabalho, uma das complicações da multimídia é como lidar com o armazenamento de informações. Uma das soluções para lidar com a grande quantidade de dados é comprimi-los em arquivos menores. Então, antes de compreender o efeito da compressão em arquivos de áudio, primeiro deve-se entender esse conceito em contextos mais simples, como na compressão de arquivos de texto.

\section{Conceito de Compressão}
A compressão de arquivos consiste no objetivo de representar informações de forma a ocupar menos espaço. No contexto da computação, diz-se que um arquivo comprimido atinge seu objetivo quando as informações de um arquivo são mantidas, mas o total de \textit{bits} do arquivo comprimido é menor que o original. Por mais simples que o conceito possa parecer, um método de compressão possui mais nuances do que ele aparenta a priori. Então, vamos destrinchar aos poucos essas questões, observando a complexidade dessa técnica.

\subsection{Para que Comprimir um Arquivo}
Uma das grandes questões no armazenamento de arquivos é o espaço que eles ocupam. Em particular no contexto de multimídia, onde imagens, sons e vídeos são misturados, o problema o espaço ocupado aumenta consideravelmente. Além disso, o transporte de informação é piorado com um aumento de dados a serem transferidos (quanto mais dados, mais tempo é gasto). Assim, a compressão de arquivos surge como uma forma de reduzir o total de espaço ocupado, aumentando o potencial de uso dos recursos e acelerando a transmissão de todas as informações.  

\subsection{Métodos de Compressão para Diferentes Contextos}
A primeira coisa a ser observada é que diferentes contextos exigem diferentes abstrações da realidade analógica para a realidade digital do computador. Consideremos dois contextos, a letra de o som de uma música. Apesar de comporem uma mesa coisa, a letra - pensando no texto em si e não em sua vocalização - possui uma natureza essencialmente diferente do som. 

\hspace{1.5 cm} Considere duas pessoas distintas, capazes de ouvir e alfabetizadas na língua portuguesa. Toda a informação contida na letra será transmitida a ambos indivíduos da mesma forma\footnote{Os gramáticos, de certo, ficarão no mínimo incomodados com tal afirmação e eles possuem razão. Duas pessoas jamais irão ler um texto da mesma forma, dadas as suas experiências particulares que compõe a formação do sujeito crítico. Contudo, o conteúdo do texto sim é o mesmo para ambas as pessoas e essa é a natureza da análise exposta.}, já os sons podem conter perdas de sua informação, devido à sua natureza estar diretamente atrelada a um sentido humano, o qual deteriora-se com o tempo. Além disso, diferentes corpos biológicos possuem variações entre si, fato este que pode implicar em uma pessoa poder ouvir mais faixas de frequência do som do que outras, aumentando a distância das informações captadas por cada um.

\hspace{1.5 cm} Dessa forma, diferentes contextos impõem uma conjuntura particular para cada caso. Perceba que um som permite uma margem para perda de informação, uma vez que existem indivíduos que sequer irão perceber essa variação caso ela esteja dentro do aceitável, enquanto um texto precisa manter a sua integridade para que todo o seu conteúdo seja devidamente transmitido. Assim, diversos algoritmos surgem para atender demandas específicas, cada um adaptado a um contexto. Ainda ancorado no som, existem formatos de áudio operam melhor para voz humana, enquanto outros buscam representar com a maior fidelidade possível a realidade. Portanto, para cada caso existe a forma ideal de realizar a compressão. 

\subsection{Compressão de Arquivos Naturais, Arquivos Harmônicos e Ruídos}
A compressão funciona por meio de algoritmos diferentes. Existem algoritmos que são específicos e outros que são genéricos. Contudo, a maior parte deles funciona de uma forma bastante semelhante: observam a realidade e tentam extrair dela padrões.

\hspace{1.5 cm} Arquivos naturais\footnote{Esse forma de referir-se à arquivos não existe na literatura, é uma notação própria deste trabalho.} são àqueles gerados naturalmente. Os textos deste documento, por exemplo, é um texto natural, ele não foi escrito seguindo um algoritmo ou utilizando uma inteligência artificial. Geralmente este tipo de arquivo possui sim padrões, contudo eles não são tão evidentes quanto arquivos artificialmente produzidos para seguir um padrão. Quando o padrão é reconhecido pelo algoritmo aquele intervalo pode ser comprimido por uma equação matemática que represente esse padrão.

\hspace{1.5 cm} Assim como os arquivos naturais, arquivos harmônicos também podem ser comprimidos, contudo o seu fator de compressão tende a ser maior. Um arquivo harmônico é gerado naturalmente a partir de estruturas harmônicas. Por exemplo, um arquivo de áudio contendo a voz de um ser humano é harmônico, uma vez que a frequência da voz é, na maior parte do tempo, constante, com algumas variações pontuais.

\hspace{1.5 cm} Em contraponto aos arquivos harmônicos existem os ruídos. Enquanto a constância caracteriza a harmonia, a aleatoriedade caracteriza um ruído. Em um arquivo de áudio, por exemplo, um ruído pode ser percebido como as vozes da plateia ou os sons do prato de uma bateria. Esse sons não possuem constância, tornando mais difícil a sua compressão (em alguns casos a quantidade de espaço ocupado pelos arquivos pode até aumentar).

\section{Análise de Compressão de Arquivos}
Com o básico da teoria de compressão explicada, podemos fazer alguns testes práticos. Em resumo, iremos comprimir alguns arquivos e avaliar os resultados.

\subsection{Materiais e Métodos}
Primeiro, desenvolveu-se um código na linguagem de programação "C". Nele, sete arquivos são gerados, cada um contendo exatos 10 MB. Esse arquivos, em formato binário, contém informações numéricas que variam de 0 a 255. Se pesarmos nele como um arquivo de texto, cada caractere é composto por \textit{8-bits}.

\hspace{1.5 cm} Esse arquivos, então foram comprimidos em dois compressores diferentes, o \textit{WinRAR} e o \textit{7-zip} $-$ no caso do primeiro os arquivos foram comprimidos duas vezes, uma no formato \textit{.zip} e outra no formato \textit{.rar}.

\hspace{1.5 cm} Quanto aos parâmetros da compressão, seguiu-se o seguinte modelo: o arquivo de controle era os parâmetros padrão dos \textit{softwares} utilizados; então outras duas compressões eram feitas, uma com a pior qualidade possível e outra com a melhor qualidade possível, esse último em casos particulares. Optou-se por assim fazer, pois, na maior parte dos casos o formato padrão de compressão já era o suficiente para reduzir drasticamente o tamanho dos arquivos; assim, não fazia sentido verificar mais casos.

\hspace{1.5 cm} Quanto às análises de resultados, optou-se por avaliar as compressões em pares. Assim será feito, pois, os arquivos gerados possuem muita proximidade entre si, tornando mais fácil uma análise dos dois em conjunto, o invés de cada um separadamente. O código pode ser visto clicando \href{https://github.com/chevitaresebruno/BSI_IFES_SERRA_FSM/tree/main}{aqui}\footnote{Não conseguimos anexar o documento no Latex e o ava só permite um arquivo por envio. Assim, estamos deixando a forma de acessá-lo por aqui mesmo: https://github.com/chevitaresebruno/BSI_IFES_SERRA_FSM/tree/main}


\section{Compressão de Ruídos e Ruídos Compactos}
Os dois primeiros arquivos gerados possuem caracteres aleatórios, nenhum critério foi escolhido para os fazer. Contudo, em um deles os caracteres variam de 0 a 255 enquanto o outros os caracteres variam de 0 a 25. Conforme explicado anteriormente, ruídos são informações de difícil compressão, onde às vezes o espaço ocupado pode até aumentar.

\hspace{1.5 cm} Informações aleatórias comportam-se como ruídos. Como os dados são desconexos entre si, sem uma ordem lógica de continuidade entre eles, a sua compressão é bastante complexa. Apesar disso, os dois arquivos diferem-se no espaço amostral existente entre eles. O primeiro arquivo, \textit{random.bin} possui caracteres que variam de 0 a 255, totalizado 256 possibilidades de caracteres, já o segundo, \textit{random_cut.bin}, possui caracteres que variam de 0 a 25, totalizando 26 possibilidades. Por mais que ambos sejam ruídos, espera-se que o primeiro tenha uma compressão bastante ruim, já o segundo deve possuir melhores taxas de compressão.

\hspace{1.5 cm} Após alguns testes, como previamente explicado, obtivemos os resultados expressos na \ref{tab:compressão_rand}. Como pode-se observar, os arquivos cut obtiveram uma taxa de compressão significativa, de ao menos 30 \%. Por outro lado, os arquivos randômicos foram bem pior comprimidos, inclusive aumentando sua taxa de compressão (apesar de ser um aumento bem pequeno, expresso pelos números negativos). Tudo dentro do esperado baseado na teoria.

\begin{table}[htbp]
    \centering
    \caption{Taxa de Compressão dos Arquivos Randômicos}
    \label{tab:compressão_rand}
    \begin{tabular}{c|c|c}
    \hline
    \textbf{Arquivos} & \textbf{Intensidade da Compressão} & \textbf{Taxa de Compressão (\%)} \\ \hline
    random\_n.zip & Normal & -0,02\% \\ \hline
    random\_n.7z & Normal & -0,01\% \\ \hline
    random\_n.rar & Normal & 0,00\% \\ \hline
    random\_b.zip & Melhor & -0,02\% \\ \hline
    random\_b.7z & Melhor & -0,01\% \\ \hline
    random\_b.rar & Melhor & 0,00\% \\ \hline
    random\_cut\_f.7z & Mais Rápido & 33,21\% \\ \hline
    random\_cut\_n.zip & Normal & 36,13\% \\ \hline
    random\_cut\_n.7z & Normal & 38,25\% \\ \hline
    random\_cut\_b.zip & Melhor & 36,13\% \\ \hline
    random\_cut\_b.7z & Melhor & 38,25\% \\ \hline
    random\_cut\_b.rar & Melhor & 34,95\% \\ \hline
    \end{tabular}
    Fonte: Autoria Própria
\end{table}

\hspace{1.5 cm} Uma observação curiosa pode ser vista na compressão dos arquivos \textit{random\_cut}. Perceba que os arquivos comprimidos pelo \textit{software 7zip}  obtveram uma melhor compressão quanto pior fosse o algoritmo de desassociação utilizado. É como se "muito ajudasse quem não atrapalhasse". Não encontramos uma resposta definitiva para isso.

\section{Compressão de PA e PG}
Agora, vamos avaliar a compressão de PAs e PGs. Antes de tudo, no entanto, devemos ressaltar a semelhança entre PAs e PGs.

\hspace{1.5 cm} Uma progressão aritmética possui múltiplas interpretações. Pode-se interpreta-lá pela notação geométrica, como a descrição de uma reta afim; bem como uma descrição algébrica, denotada por um somatório. A vantagem da interpretação geométrica é que ela resulta em uma função, facilitando a sua implementação e compreensão. A seguinte função denota uma PA, onde a_1 representa o termo inicial, r a razão e i o iésimo termo da PA:  \begin{equation*} f(i) = a_1 + (i-1) \cdot r \end{equation*}.

\hspace{1.5cm} Assim como uma PA, uma PG possui fórmula para o termo geral bastante semelhante, com algumas diferenças, mas uma estrutura muito parecida. Tanto a PA quanto a PG possuem termo inicial e uma raiz, mas o cálculo dela é diferente. Logo, é natural pensarmos que ambas as compressões serão bem próximas.  \begin{equation*} f(i) = a_1 \cdot r^{i-1} \end{equation*}.

\hspace{1.5 cm} Vamos parar de deduzir e observar os resultados da \ref{tab:compressão_papg}\footnote{A taxa de compressão foi calculada seguindo a seguinte fórmula: 1 - (tamanho comprimido / tamanho original)}. Todas as taxas de compressão foram muito satisfatórias e próximas, bem como o esperado.

\begin{table}[htbp]
    \centering
    \caption{Taxa de Compressão dos Arquivos de PA e PG}
    \label{tab:compressão_papg}
    \begin{tabular}{c|c|c}
    \hline
    \textbf{Arquivo} & \textbf{Intensidade de Compressão} & \textbf{Taxa de Compressão (\%)} \\ \hline
    pa\_n.zip & Normal & 99,90\% \\ \hline
    pa\_n.7z & Normal & 99,98\% \\ \hline
    pa\_n32.rar & Normal & 99,99\% \\ \hline
    pa\_b.zip & Melhor & 99,90\% \\ \hline
    pa\_b.rar & Melhor & 99,99\% \\ \hline
    pa\_m.7z & Melhor & 99,98\% \\ \hline
    pa\_n.7z & Normal & 99,98\% \\ \hline
    pg\_b.zip & Melhor & 99,90\% \\ \hline
    pg\_n.zip & Normal & 99,90\% \\ \hline
    pg\_b.7z & Melhor & 99,98\% \\ \hline
    pg\_n.7z & Normal & 99,98\% \\ \hline
    pg\_n.rar & Normal & 99,99\% \\ \hline
    \end{tabular}

    Fonte: Autoria Própria
\end{table}

\hspace{1.5 cm} A partir das análises, podemos verificar que o \textit{software 7zip} obteve taxas de compressões menores que o seu adversário, ação decorrente do método de compressão interno do próprio programa, uma vez que mudamos diversos parâmetros, como o tamanho do dicionário e a intensidade da compressão, mas os resultados foram os mesmos.

\subsection{Considerações Finais}
Após as análises previamente feitas, podemos chegar à algumas conclusões. A primeira delas é sobre a eficiência dos \textit{softwares} utilizados: o \textit{WinRar} provou-se superior, com as melhores taxas de compressão, onde o ideal é utilizar o formato \textit{.rar}.

\hspace{1.5 cm} Além disso, observou-se que os algoritmos são bastante competentes em compreender padrões comuns na natureza, como a composição de PAs e PGs. Por fim, ruídos realmente são problemáticos para os algoritmos. Sua compressão muitas vezes é pior que a manutenção do arquivo íntegro, sendo um problema não muito grave dependendo dos padrões de arquivos a serem comprimidos.

\captionsetup{justification=centering,margin=0cm}

%inicio do capitulo
\chapter[ÁUDIOS BINAURAIS (ATIVIDADE 2)]{ÁUDIOS BINAURAIS (ATIVIDADE 2)}
\index{ÁUDIOS BINAURAIS (ATIVIDADE 2)}

A audição humana é composta por diferentes camadas de complexidade. Como o cérebro funciona por meio de uma vasta ligação entre neurônios, um som pode despertar uma memória ou mesmo um sentimento, além de as diferenças entre os canais de som proporcionarem experiências diferentes e mais imersivas em alguns contextos.

\hspace{1.5 cm} O objetivo de um áudio binaural é proporciona a sensação de profundidade e imersão a partir de diferentes sons que são emitidos em intensidades diferentes, ou mesmo frequências diferentes, nos dois ouvidos. É importante compreender isso, pois, essa questão da \textit{mixagem} do som influencia diretamente na compressão e manipulação de formatos destes.

\hspace{1.5 cm} Essa seção é empírica e busca evidenciar as percepções dos autores deste documento de modo individual. Não existe, necessariamente, um estudo aprofundado ou científico dessas análises.

\section{Percepções de Bruno da Fonseca Chevitarese}
Um áudio binaural, como exposto anteriormente, consiste particularmente na sensação de imersão e profundidade. Como esse tipo de áudio não está associado a algo em particular, como sons da natureza, ou outras categorias, entendo que as manifestações corporais dependem de cada indivíduo. Assim, para manter um controle das observações, ater-me-ei a analisar as minhas reações em relação aos áudios propostos nas especificações do trabalho.

\hspace{1.5 cm} A partir da análise dos três áudios pude reparar que não sou muito expressivo. Mantive-me na maior parte do tempo neutro em relação aos sons, eles não me incomodavam nem afetavam de alguma forma. Contudo, sou uma pessoa com uma mente bastante imaginativa; o que eu não percebia no corpo, percebia na mente.

\hspace{1.5 cm} Reforço novamente que o áudio binaural proporciona a sensação de profundidade e imersão e em cada áudio eu me imaginava na cena. No primeiro áudio, eu percebia a movimentação do papel e tentava entender no quê o papel estava sendo "transformado", ou seja, como era amassado. Quanto ao segundo áudio tentei rastrear as bolas, compreender da onde elas vinham, a altura delas até cair no chão e a posição onde colidiram. Por fim, no terceiro áudio tentei imaginar o ambiente, compreendendo da onde as pessoas falavam e como movimentava-se pela cena. Acredito que uma parte da imersão tenha se quebrado, pois, meu inglês não é muito rebuscado, então eu tentava entender as falas sem necessariamente entendê-las.

\hspace{1.5 cm} Acredito que a minha experiência decorre das minhas experiência particulares. Eu sempre fui uma pessoa que tem a necessidade de entender exatamente o que está acontecendo e não gosto de ser pego de surpresa, então tento extrair a maior quantidade de informações possíveis do ambiente e dos corpos que o compõe

\section{Percepções de Luiz Felipe Elizeta}

Ao estar constantemente exposto à ideia de áudio binaural, adquiri um conhecimento íntimo das minhas reações a ele. Experimento arrepios e um leve sobressalto, percebo minhas emoções amplificadas, e sinto-me mais alerta e concentrado enquanto ouço. Parece quase como se estivesse sendo constantemente ludibriado, porém, paradoxalmente, também me sinto profundamente influenciado por ele, dependendo do contexto, como se fosse o principal catalisador das minhas reações instintivas.

\hspace{1.5 cm} No entanto, estes são sentimentos que já experienciei antes de estudar multimídia. Agora, descubro a existência de um áudio projetado para melhorar a concentração e o foco, e decido testá-lo. Surpreendentemente, os resultados foram bastante positivos. Durante os períodos de concentração, o áudio simplesmente se fundia ao fundo, e nos momentos mais relaxados, qualquer ruído ambiente era suprimido, permitindo-me imergir ainda mais na atividade em que estava envolvido.

\hspace{1.5 cm} Assim, acredito que algo semelhante ocorra durante o estudo: com o tempo, o áudio se torna imperceptível e os ruídos externos parecem diminuir, proporcionando um ambiente propício para a concentração.

\captionsetup{justification=centering,margin=0cm}

%inicio do capitulo
\chapter[ANALISANDO COMPRESSÃO DE ÁUDIOS (ATIVIDADE 3)]{ANALISANDO COMPRESSÃO DE ÁUDIOS (ATIVIDADE 3)}
\index{ANALISANDO COMPRESSÃO DE ÁUDIOS (ATIVIDADE 3)}
Após estudar os conceitos teóricos da compressão, é chegada a hora de verificar quais as influências dessa técnica na qualidade do áudio

\section{Definindo a Complexidade de um Áudio}
A primeira coisa a ser avaliada é como categorizar a complexidade de um aquivo de áudio. Como deve-se saber, utilizar parâmetros relativos é um problema, portanto, precisa-se definir um conceito claro e coeso de como entender um áudio simples ou complexo.

\hspace{1.5 cm} O foco deste trabalho é a análise da compressão de arquivos, assim a métrica para medir a complexidade de um arquivo será o \textit{ratio}. Esse é dado pela razão do arquivo após as modificações pelo arquivo original, segundo informações contidas no próprio material da disciplina \cite{FlavioGiraldeli}; quanto menor for o \textit{ratio} menor será a sua complexidade.

\hspace{1.5 cm} A partir dessa métrica pode-se ter uma análise menos empírica, contudo, ainda faz-se necessário o entendimento de quais características melhoram ou pioram as taxas de compressão. Diversos fatores são importantes, mas segue uma lista daqueles que mais influenciam e a sua justificativa:

\begin{itemize}
    \item Presença áudios em dois canais distintos: conforme demonstrado na secção anterior, áudios binaurais possuem informações diferentes para os dois ouvidos, caso essa diferença seja muito pequena existem algoritmos que realizam manipulações no áudio para mesclar ambos os canais, reduzindo o tamanho do arquivo;
    \item Presença de ruído: arquivos que contém menos ruídos, como plateias ao fundo ou mesmo instrumentos mais caóticos como os pratos de uma bateria, podem ser comprimidos com mais eficiência, reduzindo o \textit{ratio}; e
    \item Presença de muitas faixas de áudio em frequências muito baixas ou elevadas: a audição humana é capaz de captar sons na faixa de 20 a 20.000 hz, contudo a audição é perdida com o passar dos anos, tornando inviável para a maioria das pessoas captar essas faixas, assim, muitos compressores "cortam" essas informações dos arquivos para reduzir seus tamanhos. 
\end{itemize}

\hspace{1.5 cm} Dessa forma, concluí-se que um arquivo mais simples é aquele que contem canais de áudios com diferenças, mas com certa proximidade, pouca presença de ruído e muitas faixas de frequências altas ou baixas. Esses arquivos, ao serem comprimidos, ficarão com uma qualidade péssima, de fato, mas o objetivo não é verificar se vai ficar bom, mas sim se vai ficar compacto.

\hspace{1.5 cm} Com os critérios devidamente estabelecidos, optamos por escolher as músicas "\textit{Words}" e "\textit{It's Only Love}" como as mais simples e as músicas "Há Tempos" e "\textit{Stars}" como as mais complexas.


\section{Analisando Taxas de Compressão em Diferentes Formatos}
\hspace{1.5 cm} Após a definição das músicas mais complexas. Uma série de compressões foram feitas para avaliar as capacidades de cada compressor. Utilizando a ferramenta \textit{Foobar2000}, diferentes formatos foram utilizados e as compressões devidamente feitas. A Tabela 3 expõe cada um dos resultados.

\begin{table}[htbp]
\centering
\caption{Compressões de arquivos de áudio}
\label{tab:compressao_audio}
\begin{tabularx}{\textwidth}{|>{\centering\arraybackslash}X|*{5}{>{\centering\arraybackslash}X|}}
\hline
\textbf{Música} & \textbf{RAR (\%)} & \textbf{ZIP (\%)} & \textbf{FLAC (\%)} & \textbf{Monkey's (\%)} & \textbf{ALAC (\%)} \\
\hline
Bee Gees - Words & 66,46 & 84,89 & 39,59 & 38,28 & 39,59 \\ \hline
Bruce Springsteen - Nebraska & 70,45 & 86,29 & 50,48 & 46,50 & 50,48 \\ \hline
Coldplay - Midnight & 66,60 & 95,03 & 55,58 & 51,13 & 55,58 \\ \hline
Coldplay - Paradise & 77,83 & 96,08 & 69,03 & 64,64 & 69,03 \\ \hline
Coldplay - Sunrise & 64,52 & 90,10 & 51,44 & 46,85 & 51,44 \\ \hline
Eagles - Hotel California & 74,09 & 94,57 & 67,06 & 63,38 & 67,06 \\ \hline
Ed Sheeran - I See Fire & 68,82 & 92,86 & 55,67 & 52,62 & 55,67 \\ \hline
Embrace - Gravity & 75,95 & 95,06 & 65,79 & 60,69 & 65,73 \\ \hline
Goo Goo Dolls - We'll Be Here (When You're Gone) & 86,29 & 97,71 & 79,71 & 77,35 & 79,71 \\ \hline
John McLaughlin - Just Give It Time & 83,64 & 97,34 & 75,77 & 73,21 & 75,77 \\ \hline
Lasgo - Intro & 64,18 & 94,88 & 52,50 & 49,15 & 52,50 \\ \hline
\end{tabularx}
\end{table}

\begin{table}[htbp]
\centering
\caption{Continuaão da Tabela 3}
\begin{tabularx}{\textwidth}{|c|*{5}{>{\centering\arraybackslash}X|}}
\hline
 Músicas & \textbf{RAR (\%)} & \textbf{ZIP (\%)} & \textbf{FLAC (\%)} & \textbf{Monkey's (\%)} & \textbf{ALAC (\%)} \\
\hline
Legião Urbana - Há Tempos [Ao Vivo] & 83,84 & 97,44 & 77,91 & 74,41 & 77,91 \\ \hline
Lights Out Asia - Promontory-Cemetery & 75,48 & 96,44 & 67,20 & 62,78 & 67,20 \\ \hline
 Madonna - Vogue & 82,15 & 97,28 & 72,12 & 67,47 & 72,12 \\ \hline
Maroon 5 - Unkiss Me & 79,86 & 95,91 & 71,97 & 69,94 & 71,97 \\ \hline
Matt Cardle - It’s Only Love [Acústica] & 48,82 & 78,63 & 33,33 & 31,47 & 33,33 \\ \hline
Roxette - Stars [Almighty Remix] & 90,54 & 97,47 & 82,14 & 79,13 & 82,14 \\ \hline
Shout Out Louds - Illusions & 80,83 & 97,11 & 71,44 & 68,90 & 71,44 \\ \hline
The Album Leaf - Window & 61,29 & 88,95 & 46,86 & 38,29 & 46,86 \\ \hline
The Beatles - Yellow Submarine & 71,49 & 93,43 & 61,72 & 59,28 & 61,72 \\
\hline
\end{tabularx}
\end{table}

Como pode-se observar, o formato \textit{monkeys audio} foi o melhor em todos os testes. Além disso, os compressores genéricos foram bem menos eficientes do que os específicos. Uma das possíveis explicações para isso é que como os algoritmos são genéricos eles não atuam em cima das nuances da especificidade das músicas, enquanto os demais atuam em cima de modelos psicoacústicos, melhorando a sua eficiência.  

\subsection{Ruído Branco}
Ainda nas músicas analisada, exigiu-se a análise de um ruído branco. Como o nome sugere, o arquivo é um ruído. Após a sua compressão, o arquivo ficou maior do que sem a compressão. Conforme visto na secção 4.3, um ruído geralmente é mal comprimido, e como nesse caso, pode ser aumentado. Assim, está tudo dentro do esperado conforme estabelece a teoria.

\section{Sobre as Avaliações iniciais}
Após obtermos os dados da \ref{tab:compressao_audio}, percebemos que erramos por muito pouco. As músicas menos complexas estavam corretas, contudo, as músicas mais complexas estavam erradas. Mais especificamente, a música "Há Tempos" era a terceira mais complexa. Apesar disso, o erro foi bastante próximo do resultado real.

\begin{table}[htbp]
\centering
\caption{Complexidade das músicas em relação à compressão Monkey's}
\begin{tabularx}{\textwidth}{l|X|c}
\toprule
\textbf{Complexidade} & \textbf{Músicas} & \textbf{Monkey's (\%)} \\
\midrule
Baixa & Matt Cardle - It’s Only Love [Acústica] & 31,47\% \\
      & Bee Gees - Words                         & 38,28\% \\
      & The Album Leaf - Window                 & 38,29\% \\
      & Bruce Springsteen - Nebraska            & 46,50\% \\
      & Coldplay - Sunrise                      & 46,85\% \\
      & Lasgo - Intro                           & 49,15\% \\
\midrule
Média & Coldplay - Midnight                     & 51,13\% \\
      & Ed Sheeran - I See Fire                & 52,62\% \\
      & The Beatles - Yellow Submarine          & 59,28\% \\
      & Embrace - Gravity                       & 60,69\% \\
      & Lights Out Asia - Promontory-Cemetery   & 62,78\% \\
      & Eagles - Hotel California              & 63,38\% \\
      & Coldplay - Paradise                    & 64,64\% \\
      & Madonna - Vogue                        & 67,47\% \\
      & Shout Out Louds - Illusions            & 68,90\% \\
      & Maroon 5 - Unkiss Me                   & 69,94\% \\
\midrule
Alta  & John McLaughlin - Just Give It Time    & 73,21\% \\
      & Legião Urbana - Há Tempos [Ao Vivo]    & 74,41\% \\
      & Goo Goo Dolls - We'll Be Here (When You're Gone) & 77,35\% \\
      & Roxette - Stars [Almighty Remix]       & 79,13\% \\
\bottomrule
\end{tabularx}
\end{table}




% % ----------------------------------------------------------
% % ELEMENTOS PÓS-TEXTUAIS

\bibliography{references}

\end{document}